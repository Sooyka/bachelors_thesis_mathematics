\documentclass[a4paper]{article}
\usepackage{polski}
\usepackage[polish,english]{babel}
\usepackage[utf8]{inputenc}
\usepackage{mathtools}
\usepackage{amsfonts}
\usepackage{amsthm}
\usepackage[pdftex,
            pdfauthor={Bartosz Sójka},
            pdftitle={Praca licencjacka brudnopis},
            pdfsubject={Explanation of connection between Hopf algebras and Makov chains}]{hyperref}

%\setlength\parindent{0pt}

\newtheorem{observation}{Observation}
\newtheorem{definition}{Definition}
\newcommand{\gdd}[1]{#1^{\mathrm{gd}*}}
\newcommand{\smalltodo}[1]{\textbf{\ To do}}
\title{Explanation of connection between Hopf algebras and Markov chains}
\author{Bartosz Sójka}
\date{May 2018}

\begin{document}
\begin{align}
[z]_{/\simeq} = \left[\sum^n_{i = 1}\alpha_i(v_i,w_i)\right]_{/\simeq} &=
\displaystyle\sum^n_{i = 1}\alpha_i[(v_i,w_i)]_{/\simeq} = \\
\displaystyle\sum^n_{i = 1}\alpha_i\left[\left(\sum^{l_1}_{j=1}\gamma_{i,j}b_{i,j},
\sum^{l_2}_{k=1}\gamma_{i,k}c_{i,k}\right)\right]_{/\simeq} &=
\displaystyle\sum^n_{i = 1}\alpha_i\left[\sum^{l_1}_{j=1}\gamma_{i,j}\left(b_{i,j},
\sum^{l_2}_{k=1}\gamma_{i,k}c_{i,k}\right)\right]_{/\simeq} = \\
\displaystyle\sum^n_{i = 1}\alpha_i\left[\sum^{l_1}_{j=1}\gamma_{i,j}\left(\sum^{l_2}_{k=1}\gamma_{i,k}
\left(b_{i,j},c_{i,k}\right)\right)\right]_{/\simeq} &=
\displaystyle\sum^n_{i = 1}\alpha_i\left[\sum^{l_1}_{\substack{1 \leq j \leq l_1 \\ 1 \leq k \leq l_2}}
\gamma_{i,j}\gamma_{i,k}(b_{i,j}, c_{i,k})\right]_{/\simeq}
\end{align}

\begin{equation}
\sum^n_{\substack{1 \leq i \leq n \\1 \leq j \leq l_1 \\ 1 \leq k \leq l_2}}
\alpha_i\gamma_{i,j}\gamma_{i,k}[(b_{i,j}, c_{i,k})]_{/\simeq})_{/\simeq}
\end{equation}

Then take $\displaystyle\sum^n_{i=1}(v_i,w_i)$. There are no $v, w$
such that $ [(v,w)]_{/\simeq} = [(v_1,w_1) +\dots + (v_n,w_n)]_{/\simeq}$.
Thus for the element $[(v_1,w_1) + \dots + (v_n,w_n)]_{/\simeq}$ of $V\otimes W$ there are no $v, w$
such that $v \otimes w = [(v_1,w_1) + \dots + (v_n,w_n)]_{/\simeq}$. However, since
$[(v_1,w_1) + \dots + (v_n,w_n)]_{/\simeq} = [(v_1,w_1)]_{/\simeq} + \dots + [(v_n,w_n)]_{/\simeq}$
it can be writen as $v_1 \otimes w_1 + \dots + v_n \otimes w_n$. \\
Now we will make some futher observations on how $V \otimes W$ looks like.


 is a vector space created from
 of elemets of form $\{v \otimes w : v \in V, w \in W\}$
such that for all $v,v_1,v_2 \in V$, $w, w_1, w_2 \in W$, $k \in K$ there hold
\begin{gather*}
v \otimes w_1 + v \otimes w_2 = v \otimes (w_1 + w_2) \\
v_1 \otimes w + v_2 \otimes w = (v_1 + v_2) \otimes w \\
k (v \otimes w) = (kv) \otimes w = v \otimes (kw)
\end{gather*}
with basises $\{v_i\}_{i \in I}$, $\{w_j\}_{j \in J}$
with basis and operations defined


As algebra we will understand a vector space $\mathcal{H}$ with additional linear operation
\text{$m:\mathcal{H} \otimes \mathcal{H} \rightarrow \mathcal{H}$} called multiplication.
Natural example are polinomials. Trivial example is multimlication within a field.
Now we will introduce an important example of algebra of noncommutative variables. Later it will rise
to Hopf algebra we will be using for describing inverse riffle shuffle.



In fact this is the purpose of map $u$ to insert the copy of the field $K$ into a $K$-algebra.

("$1a$" means $a$ from $\mathcal{H}$ multiplicated by $1$ from the field $K$.)


means exactly what was in the definition of $u$.
Which mean that "counit does the same thing to comultiplication as unit to the multiplication" ;)

Since
\begin{align*}
(\Delta \otimes I)\Delta(c) &= (\Delta \otimes I)\left(\sum c_1 \otimes c_2 \right) =
\sum\Delta(c_1) \otimes c_2, \\
(I \otimes \Delta)\Delta(c) &= (I \otimes \Delta)\left(\sum c_1 \otimes c_2 \right) =
\sum c_1 \otimes \Delta(c_2).
\end{align*}

%\gebin{equation*}

$(\mathcal{H}, m, u)$

$(\mathcal{H}, \Delta, \varepsilon)$


    The canonical isomorphism between $K \otimes K$ and $K$ for all
$k_1, k_2 \in K$ taking a form $k_1 \otimes k_2 \xmapsto{\cong} k_1k_2$ will be writen as $\varphi_K$.

 and because of that we will omit it and identify $K \otimes K$ with $K$
(and, because of associativity of a field multiplication we will identify any power $K^{\otimes n}$ with $K$).
\\[8pt]

Note, that later, when there will be no risk of confusion,
we will be still using "$\cdot$" as multiplication on algebra setted by "$m$" from that algebra.

\textbf{Explanation. } In fact for a given vector space $\mathcal{H}$ with both an algebra structure
$(\mathcal{H}, m, u)$ and a coalgebra structure $(\mathcal{H}, \Delta, \varepsilon)$, $m$ and $u$
are morfisms of coalgebras iff $\Delta$ and $\varepsilon$ are morfisms of algebras and these are equivalent
to conjuction of contditions that

preceisly.


where $I^m$ means $\overbrace{I \otimes \dots \otimes I}^{m \mathrm{\ times}}$. \\

\begin{align*}
[z]_{/\simeq} = \left[\sum^n_{i = 1}\alpha_i(v_i,w_i)\right]_{/\simeq} &=
\sum^n_{i = 1}\alpha_i[(v_i,w_i)]_{/\simeq} = \\
\sum^n_{i = 1}\alpha_i\left[\left(\sum^{l_1}_{j=1}\gamma_{i,j}b_{i,j},
\sum^{l_2}_{k=1}\gamma_{i,k}c_{i,k}\right)\right]_{/\simeq} &=
\sum^n_{i = 1}\alpha_i\left[\sum^{l_1}_{j=1}\gamma_{i,j}\left(b_{i,j},
\sum^{l_2}_{k=1}\gamma_{i,k}c_{i,k}\right)\right]_{/\simeq} = \\
\sum^n_{i = 1}\alpha_i\left[\sum^{l_1}_{j=1}\gamma_{i,j}\left(\sum^{l_2}_{k=1}\gamma_{i,k}
\left(b_{i,j},c_{i,k}\right)\right)\right]_{/\simeq} &=
\sum^n_{i = 1}\alpha_i\left[\sum_{\substack{1 \leq j \leq l_1 \\ 1 \leq k \leq l_2}}
\gamma_{i,j}\gamma_{i,k}(b_{i,j}, c_{i,k})\right]_{/\simeq} = \\
\sum^n_{i = 1}\alpha_i\left(\sum_{\substack{1 \leq j \leq l_1 \\ 1 \leq k \leq l_2}}
\gamma_{i,j}\gamma_{i,k}[(b_{i,j}, c_{i,k})]_{/\simeq}\right) &=
\sum_{\substack{1 \leq i \leq n \\1 \leq j \leq l_1 \\ 1 \leq k \leq l_2}}
\alpha_i\gamma_{i,j}\gamma_{i,k}[(b_{i,j}, c_{i,k})]_{/\simeq}
\end{align*}

\begin{align*}
[z]_{/\simeq} = \left[\sum^n_{i = 1}\alpha_i(v_i,w_i)\right]_{/\simeq} &=
\sum^n_{i = 1}\alpha_i[(v_i,w_i)]_{/\simeq} \\
&= \sum^n_{i = 1}\alpha_i\left[\left(\sum^{l_1}_{j=1}\gamma_{i,j}b_{i,j},
\sum^{l_2}_{k=1}\gamma_{i,k}c_{i,k}\right)\right]_{/\simeq} \\
&= \sum^n_{i = 1}\alpha_i\left[\sum^{l_1}_{j=1}\gamma_{i,j}\left(b_{i,j},
\sum^{l_2}_{k=1}\gamma_{i,k}c_{i,k}\right)\right]_{/\simeq} \\
&= \sum^n_{i = 1}\alpha_i\left[\sum^{l_1}_{j=1}\gamma_{i,j}\left(\sum^{l_2}_{k=1}\gamma_{i,k}
\left(b_{i,j},c_{i,k}\right)\right)\right]_{/\simeq} \\
&= \sum^n_{i = 1}\alpha_i\left[\sum_{\substack{1 \leq j \leq l_1 \\ 1 \leq k \leq l_2}}
\gamma_{i,j}\gamma_{i,k}(b_{i,j}, c_{i,k})\right]_{/\simeq} \\
&= \sum^n_{i = 1}\alpha_i\left(\sum_{\substack{1 \leq j \leq l_1 \\ 1 \leq k \leq l_2}}
\gamma_{i,j}\gamma_{i,k}[(b_{i,j}, c_{i,k})]_{/\simeq}\right) \\
&= \sum_{\substack{1 \leq i \leq n \\1 \leq j \leq l_1 \\ 1 \leq k \leq l_2}}
\alpha_i\gamma_{i,j}\gamma_{i,k}[(b_{i,j}, c_{i,k})]_{/\simeq}
\end{align*}

\begin{align*}
(u\varepsilon *f)(c) &= \sum u\varepsilon(c_1) \cdot f(c_2) \\
&= \sum \varepsilon(c_1)1_A \cdot f(c_2) \\
&= \sum 1_A \cdot \varepsilon(c_1)f(c_2) \\
&= 1_A \cdot \sum \varepsilon(c_1)f(c_2) \\
&= 1_A \cdot f(c) = f(c)
\end{align*}

\begin{align*}
(f * u\varepsilon)(c) &= \sum f(c_1) \cdot u\varepsilon(c_2) \\
&= \sum f(c_1) \cdot \varepsilon(c_2)1_A \\
&= \sum f(c_1)\varepsilon(c_2) \cdot 1_A \\
&= \left(\sum f(c_1)\varepsilon(c_2)\right)\cdot 1_A \\
&= f(c) \cdot 1_A = f(c)
\end{align*}


Hence the name. \\

"with respect to coordinates" \\

in next steps \\

That means that for bialgebra $\mathcal{H}$, a linear map $S \in Hom(\mathcal{H}^C, \mathcal{H}^A)$
is an antipode iff it satisfies

\begin{align*}
\Delta(x_{i_0}\dots x_{i_k}) &= \Delta m^{[k]}(x_{i_0} \otimes \dots \otimes x_{i_k}) \\
&= (m^{[k]} \otimes m^{[k]}) \left(\sum (x_{i_0})_1 \otimes \dots \otimes (x_{i_k})_1 \otimes
(x_{i_0})_2 \otimes \dots \otimes (x_{i_k})_2\right) \\
&= \sum (x_{i_0})_1 \dots (x_{i_k})_1 \otimes
(x_{i_0})_2 \dots (x_{i_k})_2 \\
&= \sum_{S \subseteq \{i_0, \dots i_k\}} \prod_{j \in S} x_j \otimes \prod_{j \notin S} x_j.
\end{align*}

Let $\nu = (\nu_1, \dots \nu_N) \in \mathbb{N}^N$ be our deck of $N$ cards (we just pick $N$ cards from
$\mathcal{X}$ in gene

Let $\nu = (\nu_1, \dots \nu_N) \in \mathbb{N}^N$ be our deck of $N$ cards (we just pick $N$ cards from
$\mathcal{X}$ in general case they can repeat

 comes with certain probability.

 we dont know how exactly these stacks looks
like but

if we have two stacks (we will refer to them as ($L$)eft and ($R$)ight).

and we have with probability $p$ $s_1$ on $L$ and $s$ on $R$ and with probability $1-p$ $s_2$ on $L$ and
$s$ on $R$ is exactly the same situation as having

Note that for all $k_1, \dots, k_n \in K$, such that at least one of them is non-zero,
an expression $\displaystyle \sum^{n}_{i = 0}k_i ($

anallogly

Let denote that pulling apart as a $\Delta : \mathcal{H} \to \mathcal{H} \otimes \mathcal{H}$, which for all
$x_{i_0}, \dots, x_{i_n}$ gives
\begin{equation*}
\Delta(x_{i_0}\cdots x_{i_n}) = \sum_{\substack{S \subseteq
\{ i_0, \dots i_n \} \\ S = \{i_{j_1}, \dots, i_{j_l}\} \\ S^c = \{i_{k_1}, \cdots, i_{k_{n-l}} \}}}
x_{i_{j_1}}\cdots x_{i_{j_l}} \otimes x_{i_{k_1}} \cdots x_{i_{k_{n-l}}}.
\end{equation*}
For putting two piles back together by placing left on the top let us write a linear map
$m : \mathcal{H} \otimes \mathcal{H} \to \mathcal{H}$ that is concatenation, which means, that
for all $x_{i_1}, \dots, x_{i_k}, x_{j_1}, \dots, x_{j_l} \in \mathcal{X}$
\begin{equation*}
m(x_{i_1}\dots x_{i_k} \otimes x_{j_1}\dots x_{j_l}) = x_{i_1}\ldots x_{i_k}x_{j_1}\ldots x_{j_l}.
\end{equation*}

Let denote that pulling apart as a $\Delta : \mathcal{H} \to \mathcal{H} \otimes \mathcal{H}$, then for all
$s \in \mathcal{X}^*$ it will give
\begin{equation*}
\Delta(s) = \sum_{\substack{s_1 \prec s \\ s_2 = s/s_1}}
s_1 \otimes s_2.
\end{equation*}
For putting two piles back together by placing left on the top let us write a linear map
$m : \mathcal{H} \otimes \mathcal{H} \to \mathcal{H}$ that is concatenation, which means, that
for all $s_1, s_2 \in \mathcal{X}^*$
\begin{equation*}
m(s_1 \otimes s_2) = s_1s_2.
\end{equation*}
 all $s = x_{i_1}\dots x_{i_n} \in \mathcal{X}^*$
that meet the condition:

\begin{equation*}
\mathbb{P}{F_{n+1} = s_2 \mid F_n = s_1 } = \mathbb{I_{n+1} = s_1 \mid I_n = s_2}.
\end{equation*}

, we percive it as a certain
amount of stacks

$\backslash\backslash$
Indeed it turned out that a vector space of

At this point we can start to think that it is convinient to interprete a space of possible arragements
of two stacks of cards as a $\mathcal{H} \otimes \mathcal{H}$. \\[4pt]

So we need a vector space with basis made of pairs of basis vectors from $\mathcal{H}$ with actions on them
that are linear to both coordinates. A tensor product $\mathcal{H} \otimes \mathcal{H}$
is the less-degenerated vector space with that properties.\\[4pt]

We will now show, that what we just had defined indeed give the same results as standard model of inverse
riffle shuffling.
In inverse riffle shuffling

Which also equvalent to that matrixes of $(F_i)_{i \geq 0}$ and $(I_i)_{i \geq 0}$ are transpositions of
each other.

a $\backslash$ big! $\backslash$ huge! $\backslash$ giant!

We will be working of an examples of riffle shuffle and inverse riffle shuffle cards shuffling as
our Markov chains. \\
How we will put them in the algebraic way? \\
First we will do this with inverse riffle shuffle, the forward riffle shuffle will then apear in a natural
way. \\


Earlier it was said that comultiplication can be sometimes viewed as a sum
of possible divisions into smaller objects. It happends naturally when we are working with graded bialgebras.
Like in example of polinomials, where natural grading is by degree. \\

\subsubsection{Graded, connected Hopf algebra of non-commuting variables}
This is a main example of our interest. \\
Let K be a field with characteristic 0.
Let $\mathcal{X} = \{x_1, \dots, x_N\}$ be a finite set. For every $n \in \mathbb{N}$ let
$H_n$ be a vector space having as a basis all words of length $n$ made of elements
of $\mathcal{X}$. Let $\mathcal{H} \coloneqq \displaystyle\bigoplus^{\infty}_{i = 0} \mathcal{H}_i$.
Let $m : \mathcal{H} \otimes \mathcal{H} \to \mathcal{H}$ be concatenation of words,
that is, for all $k, l \in \mathbb{N}$
for all $x_{i_0}, \dots, x_{i_k}, x_{j_0}, \dots, x_{j_l} \in \mathcal{X}$
\begin{equation*}
m(x_{i_0}\dots x_{i_k} \otimes x_{j_0}\dots x_{j_l}) \coloneqq
x_{i_0}\ldots x_{i_k}x_{j_0}\ldots x_{j_l}.
\end{equation*}
Let $\Delta : \mathcal{H} \to \mathcal{H} \otimes \mathcal{H}$ be defined for all elements from
$\mathcal{X}$ as
\begin{equation*}
\Delta(x_i) = x_i \otimes 1_\mathcal{H} + 1_\mathcal{H} \otimes x_i.
\end{equation*}
and extends lineary and multiplically \\
\textbf{Lemma. } Then $\mathcal{H}$ is the a graded, connected Hopf algebra that is cocomutative.
\begin{proof}
Associativity of $m$ and coassociativity of $\Delta$ are obvious. Actions fit together,
because we define them so. Algebra is graded and connnected because it is.
\end{proof}
We can see $\Delta(x_{i_0}\dots x_{i_k})$ how looks like:
\begin{align*}
\Delta(x_{i_0}\dots x_{i_k}) &= \Delta m^{[k]}(x_{i_0} \otimes \dots \otimes x_{i_k}) \\
&= (m^{[k]} \otimes m^{[k]}) \left(\sum (x_{i_0})_1 \otimes \dots \otimes (x_{i_k})_1 \otimes
(x_{i_0})_2 \otimes \dots \otimes (x_{i_k})_2\right) \\
&= \sum (x_{i_0})_1 \dots (x_{i_k})_1 \otimes
(x_{i_0})_2 \dots (x_{i_k})_2 \\
&= \sum_{S \subseteq \{ i_0, \dots i_k \} } \prod_{j \in S} x_j \otimes \prod_{j \notin S} x_j.
\end{align*}
where $S$ is a multiset, because some of the $i_0, \dot i_k$ can be the same.
Form of that coproduct is like that because when for all $x_i \in \mathcal{X}$ we have
$\Delta(x_i) = x_i \otimes 1_\mathcal{H} + 1_\mathcal{H} \otimes x_i$ then \\[8pt]

a ! \large! \Large! \huge! \Huge! \normalsize

\section{Summarize}
$\displaystyle\sum\ \bigoplus\ +\ \oplus$

\section{Productise}
$\displaystyle\prod\ \bigotimes\ \bigodot\ \times\ *$
\hfill \break
To jest źle: \\
Please note, that at first it can be not clear that $h^*m \in \mathcal{H}^* \otimes \mathcal{H}^*$.
For sure $h^* \in (\mathcal{H} \otimes \mathcal{H})^*$ but in general case we have
$ \mathcal{H}^* \otimes \mathcal{H}^* \subseteq (\mathcal{H} \otimes \mathcal{H})^*$ (with inclusion given by
canonical injection) but not necessarily
$ \mathcal{H}^* \otimes \mathcal{H}^* = (\mathcal{H} \otimes \mathcal{H})^*$.
Although we are in graded bialgebra $\mathcal{H}$ and because of that we know that
$h^*m \in \displaystyle\left(\bigoplus^\infty_{n=0}\mathcal{H}_n \otimes \mathcal{H}_n\right)^* =
\displaystyle\bigoplus^\infty_{n = 0} (\mathcal{H}_n \otimes \mathcal{H}_n)^*$. Futher, for all
$n \in \mathbb{N}$ since $\mathcal{H}_n$ is finite-dimentional we have that
$(\mathcal{H}_n \otimes \mathcal{H}_n)^* = \mathcal{H}_n^* \otimes \mathcal{H}_n^*$.
Hence
\begin{equation*}
h^*m \in \displaystyle\bigoplus^\infty_{n = 0} \mathcal{H}_n^* \otimes \mathcal{H}^*_n =
\mathcal{H}^* \otimes \mathcal{H}^*.
\end{equation*}

Now we will describe how a dual algebra to $\mathcal{H}$ looks like. \\
Let $\mathcal{H}^*$ be the dual vector space to $\mathcal{H}$. (A space of linear functions from
$\mathcal{H}$ to $K$). We define multiplication
$\Delta^* : \mathcal{H}^* \otimes \mathcal{H}^* \to \mathcal{H}^*$ and comultiplication
$m^* : \mathcal{H}^* \to \mathcal{H}^* \otimes \mathcal{H}^*$ as
(for all $h_1^*, h_2^*, h^* \in \mathcal{H}^*$):
\begin{align*}
\Delta^*(h_1^* \otimes h_2^*) = (h_1^* \otimes h_2^*)\Delta, \\
m^*(h^*) = h^*m.
\end{align*}

Structure detail described in this paragraph will be important in chapter about eigenbasises.
[GR89] shows that symmetrized sums of certain primitive elements form basis of a free associative algebra.
It will turn out that this will be left eigenbasis of $m\Delta$. Now we will introduce methods for
construction of that basis. Explanation why this is an eigenbasis will came in Chapter 4. This whole
paragraph is exactly the same as in ~\cite{Diaconis2014} I put it here because it is quite short and \\

\begin{align*}
\Psi^{[i]}(h) = m^{[i]}\Delta^{[i]}(h) &= \\
m^{[i]}(\sum h_1 \otimes \dots \otimes h_i) &= \\
m^{[i]}(\underbrace{\overbrace{h \otimes 1_\mathcal{H} \otimes \dots \otimes 1_\mathcal{H}}^{i\
\mathrm{factors}} +
\overbrace{1_\mathcal{H} \otimes h \otimes \dots \otimes 1_\mathcal{H}}^{i\ \mathrm{factors}} +
\dots +
\overbrace{1_\mathcal{H} \otimes 1_\mathcal{H} \otimes \dots \otimes h}^{i\ \mathrm{factors}}}_{i\
\mathrm{summands}} ) &= \\
\underbrace{m^{[i]}(\overbrace{h \otimes 1_\mathcal{H} \otimes \dots \otimes 1_\mathcal{H}}^{i\
\mathrm{factors}}) +
m^{[i]}(\overbrace{1_\mathcal{H} \otimes h \otimes \dots \otimes 1_\mathcal{H}}^{i\ \mathrm{factors}}) +
\dots +
m^{[i]}(\overbrace{1_\mathcal{H} \otimes 1_\mathcal{H} \otimes \dots \otimes h}^{i\ \mathrm{factors}})}_{i\
\mathrm{summands}} &=
\end{align*}

For simplifying the notation we will write a symbol for algebra multiplication also for componentwise
multiplication, so for all ${^1h}, \dots, {^nh} \in \mathcal{H}$:
\begin{equation}
\Delta ^{[m]}m^{[n]}({^1h}\otimes \dots\otimes {^nh}) =  \eqqcolon \sum
\end{equation}

\begin{proof}
\begin{align*}
\Psi^{[a]}(\sum_{\sigma \in S_k} x_{\sigma(1)}\cdot\ldots\cdot x_{\sigma(k)}) &= \\
m^{[a]}\Delta^{[a]}(\sum_{\sigma \in S_k} x_{\sigma(1)}\cdot\ldots\cdot x_{\sigma(k)}) &= \\
m^{[a]}(\sum_{\sigma \in S_k} \Delta^{[a]}(x_{\sigma(1)}\cdot\ldots\cdot x_{\sigma(k)})) &= \\
m^{[a]}(\sum_{\sigma \in S_k} (x_{\sigma(1)}\otimes 1_\mathcal{H} + 1_\mathcal{H} \otimes x_{\sigma(1)}) \cdot
\ldots \cdot (x_{\sigma(k)} \otimes 1_\mathcal{H} + 1_\mathcal{H} \otimes x_{\sigma(k)}) &=
\end{align*}
\end{proof}

\begin{align*}
\Delta([x, y]) = \Delta(x\cdot y - y\cdot x) = \Delta (x\cdot y) - \Delta(y\cdot x) =
\Delta m(x \otimes y) - \Delta m(y \otimes x) =  \sum x_1 \cdot y_1 \otimes x_2 \cdot y_2 - \sum y_1
\cdot x_1 \otimes y_2 \cdot x_1 =
\end{align*}

\begin{proof} (of the lemma) \\
Proof will be as follows: \\
- we will show that if \\
- then we will show that $w \in \mathcal{H}_\nu$ iff $\lambda(l_1)\dots\lambda(l_k) \in \mathcal{H}_\nu$,
where $(l_1,\dots,l_k)$ is Lyndon factorisation of $w$.
\end{proof}

Let $|w|$ be the length of word $w$. For a word $w = a_1\dots a_{|w|}$ and permutation
$\sigma \in S_{|w|}$ let $\sigma(w) \coloneqq a_{\sigma(1)}\dots a_{\sigma(|w|)}$. \\ Let $\sim$ be
a relation on $\mathcal{X}^* \times \mathcal{X}^*$ such that for all $w, v \in \mathcal{X}^*$ such that $w = a_1\dots a_{|w|}$, $v = b_1\dots b_{|v|}$ for
$a_1,\dots, a_{|w|}, b_1, \dots, b_{|v|} \in \mathcal{X}$
\begin{equation*}
w \sim v \iff |w| = |v| = n \land \exists_{\sigma \in S_n} a_1\dots a_n = b_{\sigma(1)}\dots b_{\sigma(n)}
\end{equation*}

\begin{proof}
For the forward riffle shuffle we want to every possible permutation have probability 1 exept of identity
with probability $n-1$. Reachable permutations are the same because of form of actions (just the same)
we litterally cut that deck and put in on the top. coefficients holds, because numbers of the number of
occurences holds. Therefore its indeed the inverse riffle shuffling.
\end{proof}

\subsubsection{Alternative structure}
\indent Now we will describe an hopf algebra structure on $\mathcal{H}^{\mathrm{gd}*}$
{(definition $\gdd{V}$ for a given $V$ can be found in 2.1.1)}. \\
We define multiplication
$\Delta^* : \mathcal{H}^{\mathrm{gd}*} \otimes \gdd{\mathcal{H}} \to \gdd{\mathcal{H}}$ and
comultiplication
\text{$m^* : \gdd{\mathcal{H}} \to \gdd{\mathcal{H}} \otimes \gdd{\mathcal{H}}$} as
(for all $h_1^*, h_2^*, h^* \in \gdd{\mathcal{H}}$):
\begin{align*}
\Delta^*(h_1^* \otimes h_2^*) = (h_1^* \otimes h_2^*)\Delta, \\
m^*(h^*) = h^*m.
\end{align*}

\subsubsection{Alternative structure \smalltodo{a}}
\indent Now we will describe an alternative Hopf algebra structure on $\mathcal{H}$ - a vector space spanned
by finite words over fixed alphabet $\mathcal{X}$. It will be the
structure of $\displaystyle\bigoplus^{\infty}_{i = 0} \mathcal{H}_i^*$ with actions induced by actions from
Hopf algebra $\mathcal{H}$ so firstly we will introduce that structure on
$\displaystyle\bigoplus^{\infty}_{i = 0} \mathcal{H}_i^*$. But since $\mathcal{H} =
\displaystyle\bigoplus^{\infty}_{i = 0} \mathcal{H}_i$ and $\displaystyle\bigoplus^{\infty}_{i = 0}
\mathcal{H}_i \simeq \displaystyle\bigoplus^{\infty}_{i = 0} \mathcal{H}_i^*$ as a linear spaces, it will
work on $\mathcal{H}$ the same. We will denote a Hopf algebra with this structure
as $\mathcal{H}^*$ and call it a graded dual to $\mathcal{H}$. (but it will NOT be isomorphic to
$\mathcal{H}^*$ - vector space dual to $\mathcal{H}$). (in~\cite{Diaconis2014} that Hopf algebra is also
denoted as $\mathcal{H}^*$).  \\
It will turn out that it describes the structure of forward riffle shuffle. \\
\indent Let denote $\gdd{\mathcal{H}}$ for $\displaystyle\bigoplus^{\infty}_{i = 0} \mathcal{H}_i^*$.
We define multiplication
$\Delta^* : \mathcal{H}^{\mathrm{gd}*} \otimes \gdd{\mathcal{H}} \to \gdd{\mathcal{H}}$ and
comultiplication
\text{$m^* : \gdd{\mathcal{H}} \to \gdd{\mathcal{H}} \otimes \gdd{\mathcal{H}}$} as
(for all $h_1^*, h_2^*, h^* \in \gdd{\mathcal{H}}$):
\begin{align*}
\Delta^*(h_1^* \otimes h_2^*) &= (h_1^* \otimes h_2^*)\Delta, \\
m^*(h^*) &= h^*m.
\end{align*}

\begin{gather*}
\sum g_1 \cdot h_1 \otimes g_2 \cdot h_2 = \\
\sum \{g_1 \cdot h_1 \otimes g_2 \cdot h_2 : g_1g_2=g \mathrm{\ and\ } h_1h_2 = h\} =\\
\sum \{ \sum\{k : g_1 \prec k \mathrm{\ and\ } h_1 = k/g_1\} \otimes
\sum\{k : g_2 \prec k \mathrm{\ and\ } h_2 = k/g_2\} : g_1g_2=g \mathrm{\ and\ } h_1h_2 = h\} = \\
\sum\{ \sum\{ m_1 \otimes m_2 : k=m_1m_2 \} : g \prec k \mathrm{\ and\ } h = k/g\} = \\
\sum\{m^*(k) : g \prec k \mathrm{\ and\ } h = k/g\} = \\
m^*\left(\sum\{k : g \prec k \mathrm{\ and\ } h = k/g\}\right) = \\
m^*\Delta^*(g \otimes h)
\end{gather*}
interlace is different
leads to a specific term

The pair of interlace and division from $m^*\Delta^*$ generates divisions of $g$ and $h$ as "that letters
that went to the prefix" and "that letters that went to the suffix" and interlaces of that prefixes and suffixes of $g$ and $h$ that are primal interlace restricted to a part of word. \\
Having
Now we will see that there is another method of introducing that structure
The
structure of $\displaystyle\bigoplus^{\infty}_{i = 0} \mathcal{H}_i^*$ with actions induced by actions from
Hopf algebra $\mathcal{H}$ (let $\gdd{\mathcal{H}} \coloneqq \displaystyle\bigoplus^{\infty}_{i = 0} \mathcal{H}_i^*$): \\
multiplication
$\Delta^* : \mathcal{H}^{\mathrm{gd}*} \otimes \gdd{\mathcal{H}} \to \gdd{\mathcal{H}}$ and \\
comultiplication
\text{$m^* : \gdd{\mathcal{H}} \to \gdd{\mathcal{H}} \otimes \gdd{\mathcal{H}}$} \\
such that for all $h_1^*, h_2^*, h^* \in \gdd{\mathcal{H}}$:
\begin{align*}
\Delta^*(h_1^* \otimes h_2^*) &= (h_1^* \otimes h_2^*)\Delta, \\
m^*(h^*) &= h^*m.
\end{align*}
so firstly we will introduce that structure on
$\displaystyle\bigoplus^{\infty}_{i = 0} \mathcal{H}_i^*$. But since $\mathcal{H} =
\displaystyle\bigoplus^{\infty}_{i = 0} \mathcal{H}_i$ and $\displaystyle\bigoplus^{\infty}_{i = 0}
\mathcal{H}_i \simeq \displaystyle\bigoplus^{\infty}_{i = 0} \mathcal{H}_i^*$ as a linear spaces, it will
work on $\mathcal{H}$ the same. We will denote a Hopf algebra with this structure
as $\mathcal{H}^*$ and call it a graded dual to $\mathcal{H}$. (but it will NOT be isomorphic to
$\mathcal{H}^*$ - vector space dual to $\mathcal{H}$). (in~\cite{Diaconis2014} that Hopf algebra is also
denoted as $\mathcal{H}^*$).  \\
It will turn out that it describes the structure of forward riffle shuffle. \\
\indent Let denote $\gdd{\mathcal{H}}$ for $\displaystyle\bigoplus^{\infty}_{i = 0} \mathcal{H}_i^*$.
We define multiplication
$\Delta^* : \mathcal{H}^{\mathrm{gd}*} \otimes \gdd{\mathcal{H}} \to \gdd{\mathcal{H}}$ and
comultiplication
\text{$m^* : \gdd{\mathcal{H}} \to \gdd{\mathcal{H}} \otimes \gdd{\mathcal{H}}$} as
(for all $h_1^*, h_2^*, h^* \in \gdd{\mathcal{H}}$):
\begin{align*}
\Delta^*(h_1^* \otimes h_2^*) &= (h_1^* \otimes h_2^*)\Delta, \\
m^*(h^*) &= h^*m.
\end{align*}

so firstly we will introduce that structure on
$\displaystyle\bigoplus^{\infty}_{i = 0} \mathcal{H}_i^*$. But since $\mathcal{H} =
\displaystyle\bigoplus^{\infty}_{i = 0} \mathcal{H}_i$ and $\displaystyle\bigoplus^{\infty}_{i = 0}
\mathcal{H}_i \simeq \displaystyle\bigoplus^{\infty}_{i = 0} \mathcal{H}_i^*$ as a linear spaces, it will
work on $\mathcal{H}$ the same. We will denote a Hopf algebra with this structure
as $\mathcal{H}^*$ and call it a graded dual to $\mathcal{H}$. (but it will NOT be isomorphic to
$\mathcal{H}^*$ - vector space dual to $\mathcal{H}$). (in~\cite{Diaconis2014} that Hopf algebra is also
denoted as $\mathcal{H}^*$).  \\
It will turn out that it describes the structure of forward riffle shuffle. \\

Here we will provide a more specific probabilistic interpretation of spaces and actions in $\mathcal{H}$.
We will do so by introduce algebraic structure on inverse riffle shuffle step-by-step. What we will end
up with will be eventually exactly the non-commuting algebra. Note, that what is written below are only
some of observations how structures of free-assiociative algebra and inverse riffle shuffle Markov chain
works togrther. It will not be proof that these structures are litterally the same nor that these arguments
apply  in general to all Hopf algebras and Markov chains.\\[4pt]


\end{document}
